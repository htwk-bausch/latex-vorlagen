\documentclass[
  a4paper,	      % DIN A4
	11pt,           % Globale Schriftgröße
  parskip=half*,  % Nach Absatz kein Einrücken
]{scrartcl}
\usepackage{../common/htwk}  % Veränderungen am Dokument (siehe htwk.sty)

\begin{document}

% Titel (erst nach \begin{document}, damit babel bereits voll aktiv ist:
%\titlehead{\small{Hochschule für Technik, Wirtschaft und Kultur (HTWK) Leipzig}}% optional
%\subject{Art des Dokuments} % optional
\title{Titel}  % obligatorisch
\subtitle{Arbeitsplan für das Praxisforschungsprojekt}     % optional
\author{Vorname Name}                                      % obligatorisch
%\author{Vorname Name (Matrikel-Nummer: xxxxx)}            % obligatorisch
\date{\today} % sinnvoll
\maketitle % verwendet die zuvor gemachte Angaben zur Gestaltung eines Titels

\section{Einführung}
Dieses Dokument soll zeigen, dass Sie sich bereits mit der Thematik, die Sie im Rahmen eines Praxisprojekts bearbeiten möchten, auseinandergesetzt haben. Dazu zählt zum einen, dass Sie die Aufgabenstellung gelesen und verstanden haben und den Aufgabenkomplex in Einzelprobleme zerlegt haben. Am Ende sollten Sie wissen, welche \glqq Baustellen\grqq{} Sie bearbeiten müssen und wo sich eventuell der Teil der Arbeit verbirgt, der die größten Herausforderungen mit sich bringt.

In der Vorbereitung ist es auch immer hilfreich, sich mit dem aktuellen Stand von Technik und Wissenschaft vertraut zu machen. Lesen Sie dazu wissenschaftliche Aufsätze, die Ihnen Ihr Betreuer zur Verfügung gestellt hat oder die Sie zum Thema selbst recherchiert haben. Je klarer ihre Vorstellungen von der zu lösenden Aufgabe am Anfang der Arbeit ist, desto schneller kommen Sie zum Ziel oder können frühzeitig entsprechende Hürden identifizieren, die sich gemeinsam mit ihre/m Betreuer/in diskutieren oder lösen lassen.

Umreißen Sie deshalb in der Einführung die Problemstellung mit eigenen Worten und formulieren Sie eine Motivation, warum dieses Problem gelöst werden muss.

Ziel ist es, dem Leser knapp zu vermitteln, was das Problem ist und wie ein möglicher Lösungsansatz aussehen kann.

\section{Ziele des Praxisforschungsprojekts}

Beschreiben Sie die Ziele, die Sie im Projekt erreichen möchten und skizzieren Sie einen groben Arbeitsplan. Benennen Sie einzelne Arbeitspakete, die Sie durchführen möchten und welcher Zeitraum dafür eingeplant ist.

Hier ein Beispiel:

\begin{itemize}\tightlist
  \item Recherche zu Hardware X und Software Y (ca. 2 Wochen)
  \item Einarbeitung und Implementierung von Algorithmus XYZ (ca. 3 Wochen)
  \item Dokumentation (ca. 2 Wochen)
\end{itemize}

Gegebenenfalls können hier auch Rahmenbedingungen genannt werden, die für die Arbeit notwendig sind. Gleichzeitig können Sie auf bereits veröffentlichte Ergebnisse verweisen, die Ihnen bei der Lösungsfindung einen ersten Ansatz geben.

\subsection{Teilziel 1}
Werden Teilziele etwas umfangreicher beschrieben, können auch Unterabschnitte angelegt werden, um eine bessere Übersichtlichkeit zu erzielen. Das ergibt natürlich nur dann Sinn, wenn es mehrere Teilziele im Projekt gibt.

\subsection{Teilziel 2}

\section{Zusammenfassung und Ausblick}
Hierhin gehört ein erstes Fazit und ggf. der Ausblick auf weitere Dinge, die im Rahmen des Projekts umgesetzt werden sollen.

Dient das Praxisforschungsprojekt als Vorarbeit für eine anschließende Bachelor- oder Masterarbeit, kann hier aufgezeigt werden, was das Ergebnis dieser Vorarbeit ist und wo die anschließende Arbeit weitergeführt wird.

\begin{thebibliography}{1}

\bibitem{latexcompanion}Michel Goossens, Frank Mittelbach, and Alexander Samarin. \textit{The \LaTeX\ Companion}. Addison-Wesley, Reading Massachusetts, 1993.

\bibitem{einstein} Albert Einstein. \textit{Zur Elektrodynamik bewegter K{\"o}rper}. (German) [\textit{On the electrodynamics of moving bodies}. Annalen der Physik, 322(10):891–921, 1905.

\bibitem{knuthwebsite} Knuth: Computers and Typesetting,\\ \texttt{http://www-cs-faculty.stanford.edu/\~{}uno/abcde.html}
\end{thebibliography}

\end{document}

\end{document}
